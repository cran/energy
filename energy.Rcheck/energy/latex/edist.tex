\Header{edist}{E-distance}
\keyword{multivariate}{edist}
\keyword{cluster}{edist}
\keyword{nonparametric}{edist}
\begin{Description}\relax
Returns the E-distances (energy statistics) between clusters.
\end{Description}
\begin{Usage}
\begin{verbatim}
 edist(x, sizes, distance=FALSE, ix = 1:sum(sizes))
\end{verbatim}
\end{Usage}
\begin{Arguments}
\begin{ldescription}
\item[\code{x}] data matrix of pooled sample or Euclidean distances
\item[\code{sizes}] vector of sample sizes
\item[\code{distance}] logical: if TRUE, x is a distance matrix
\item[\code{ix}] a permutation of the row indices of x 
\end{ldescription}
\end{Arguments}
\begin{Details}\relax
A vector containing the pairwise two-sample multivariate 
\eqn{\mathcal{E}}{E}-statistics for comparing clusters or samples is returned. 
The e-distance between clusters is computed from the original pooled data, 
stacked in matrix \code{x} where each row is a multivariate observation, or 
from the distance matrix \code{x} of the original data, or distance object 
returned by \code{dist}. The first \code{sizes[1]} rows of the original data 
matrix are the first sample, the next \code{sizes[2]} rows are the second 
sample, etc. The permutation vector \code{ix} may be used to obtain
e-distances corresponding to a clustering solution at a given level in
the hierarchy.

The e-distance between two clusters \eqn{C_i, C_j}{}
of size \eqn{n_i, n_j}{} 
proposed by Szekely and Rizzo (2003ab)
is the e-distance \eqn{e(C_i,C_j)}{}, defined by
\deqn{e(C_i,C_j)=\frac{n_i n_j}{n_i+n_j}[2M_{ij}-M_{ii}-M_{jj}],
}{e(S_i, S_j) = (n_i n_j)(n_i+n_j)[2M_(ij)-M_(ii)-M_(jj)],}
where
\deqn{M_{ij}=\frac{1}{n_i n_j}\sum_{p=1}^{n_i} \sum_{q=1}^{n_j}
\|X_{ip}-X_{jq}\|,}{M_{ij} = 1/(n_i n_j) sum[1:n_i, 1:n_j] ||X_(ip) - X_(jq)||,}
\eqn{\|\cdot\|}{|| ||} denotes Euclidean norm, and \eqn{X_{ip}}{X_(ip)} denotes the p-th observation in the i-th cluster.\end{Details}
\begin{Value}
A object of class \code{dist} containing the lower triangle of the
e-distance matrix of cluster distances corresponding to the permutation 
of indices \code{ix} is returned.\end{Value}
\begin{Author}\relax
Maria Rizzo \email{rizzo@math.ohiou.edu}
\end{Author}
\begin{References}\relax
Szekely, G. J. and Rizzo, M. L. (2003a) Hierarchical Clustering
via Joint Between-Within Distances, submitted.

Szekely, G. J. and Rizzo, M. L. (2003b) Testing for Equal
Distributions in High Dimension, submitted.

Szekely, G. J. (2000) \eqn{\mathcal{E}}{E}-statistics: Energy of 
Statistical Samples, preprint.\end{References}
\begin{SeeAlso}\relax
\code{\Link{energy.hclust}}
\code{\Link{eqdist.etest}} \code{\Link{ksample.e}}
\end{SeeAlso}
\begin{Examples}
\begin{ExampleCode}
 ## compute e-distances for 3 samples of iris data
 data(iris)
 edist(iris[,1:4], c(50,50,50))

 ## compute e-distances from a distance object
 data(iris)
 edist(dist(iris[,1:4]), c(50, 50, 50), distance=TRUE)

 ## compute e-distances from a distance matrix
 data(iris)
 d <- as.matrix(dist(iris[,1:4]))
 edist(d, c(50, 50, 50), distance=TRUE) 
\end{ExampleCode}
\end{Examples}

