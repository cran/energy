\Header{mvnorm.e}{E-statistic (Energy Statistic) for Testing Multivariate Normality}
\keyword{multivariate}{mvnorm.e}
\keyword{htest}{mvnorm.e}
\begin{Description}\relax
Computes the E-statistic (energy statistic) for testing multivariate 
or univariate normality when parameters are estimated.
\end{Description}
\begin{Usage}
\begin{verbatim}
mvnorm.e(x)
\end{verbatim}
\end{Usage}
\begin{Arguments}
\begin{ldescription}
\item[\code{x}] matrix or vector of sample data
\end{ldescription}
\end{Arguments}
\begin{Details}\relax
If \code{x} is a matrix, each row is a multivariate observation. The
data will be standardized to zero mean and identity covariance matrix
using the sample mean vector and sample covariance matrix. If \code{x}
is a vector, the univariate statistic \code{normal.e(x)} is returned. 
If the data contains missing values or the sample covariance matrix is 
singular, NA is returned.

The \eqn{\mathcal{E}}{E}-test of multivariate normality was proposed
and implemented by Szekely and Rizzo (2004). The test statistic for 
d-variate normality is given by
\deqn{\mathcal{E} = n (\frac{2}{n} \sum_{i=1}^n E\|y_i-Z\| - 
E\|Z-Z'\| - \frac{1}{n^2} \sum_{i=1}^n \sum_{j=1}^n \|y_i-y_j\|),
}{E = n((2/n) sum[1:n] E||y_i-Z|| - E||Z-Z'|| - (1/n^2) sum[1:n,1:n]
||y_i-y_j||),}
where \eqn{y_1,\ldots,y_n}{} is the standardized sample, 
\eqn{Z, Z'}{} are iid standard d-variate normal, and
\eqn{\| \cdot \|}{|| ||} denotes Euclidean norm.\end{Details}
\begin{Value}
The value of the \eqn{\mathcal{E}}{E}-statistic for multivariate
(univariate) normality is returned.\end{Value}
\begin{Author}\relax
Maria Rizzo \email{rizzo@math.ohiou.edu}
\end{Author}
\begin{References}\relax
Szekely, G. J. and Rizzo, M. L. (2004) A New Test for 
Multivariate Normality, \emph{Journal of Multivariate Analysis},
to appear.

Rizzo, M. L. (2002). A New Rotation Invariant Goodness-of-Fit Test,
Ph.D. dissertation, Bowling Green State University.

Szekely, G. J. (1989) Potential and Kinetic Energy in Statistics, 
Lecture Notes, Budapest Institute of Technology (Technical University).\end{References}
\begin{SeeAlso}\relax
\code{\Link{normal.e}}
\end{SeeAlso}
\begin{Examples}
\begin{ExampleCode}
 
 ## compute multivariate normality test statistic for iris Setosa data
 data(iris)
 mvnorm.e(iris[1:50, 1:4])
 \end{ExampleCode}
\end{Examples}

